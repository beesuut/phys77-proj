\documentclass{notes}
\graphicspath{{../Images/}}

\fancyhead[l]{Capstone Project}
\fancyhead[c]{Physics 77}
\fancyhead[r]{December 3, 2024}
\fancyfoot[c]{Page \thepage\ of \pageref{LastPage}}

\begin{document}

\section{What is Our Project?}

\begin{enumerate}
    \item Simulating a nuclear reactor using monte-carlo
    \subitem Repeating random results to find a statistical output
    \subitem \indicates No time dependence: actions happen in `ticks'
\end{enumerate}

\subsection{Assumptions}

\begin{enumerate}
    \item Reactive elements are in a homogenous mixture
    \item Spherical reactor
    \item Fissions do not release new neutrons
    \subitem For simplicity: This event would cause a never-ending cycle of reactions because the fuel is assumed to be homogenous and unchanging in quantity
\end{enumerate}

\subsection{Inputs}

\begin{enumerate}
    \item Reactor Size
    \item Neutrons
    \subitem Position, Velocity, Energy (?)
    \item Cross-sections
\end{enumerate}

\subsubsection{Cross-Sections}

A varying `size' dependent on neutron energy and reactive element

\indicates Determines probability of an event occurring

Cross-sections are different for each event

\subsubsection{Possible Events (And How They Are Processed)}

Fission: The neutron collides with an atom with enough energy to create fission

\tab \indicates The fission event is counted and the neutron is removed from the simulation

Escape: The netron escapes the reactor

\tab \indicates The escape event is counted and the neutron is removed from the simulation

Absorption (Inelastic Collision): The neutron collides with an atom and sticks

\tab \indicates The absorption event is counted and the neutron is removed from the simulation

Scattering (Elastic Collision): The neutron bounces off an atom (and loses energy?)

\tab \indicates The scattering event is counted and the neutron is run through the simulation recursively until escape (with lower energy?)

\subsection{Outputs}

Number of each event

\begin{enumerate}
    \item Fissions: Energy released, productivity of reactor
    \item Escapes
    \item Absorptions
    \item Scattering
\end{enumerate}

\section{Project Flowchart}

Initial conditions entered:

\begin{enumerate}
    \item Reactor Radius
    \item Cross Sections
    \item Number of Neutrons
\end{enumerate}

Neutrons `spawned' with a position, direction, and energy

Time `ticked,' moving neutrons by an amount dictated by density function in the initialized direction

Check what neutron does and perform recursion as necessary

\section{Our Process}

\subsection{Starting}

We began with simulating a single neutron to test procedures

Issues:

\begin{enumerate}
    \item Position and direction each had 3 separate variables
    \item When neutron was randomly spawned, it was not always inside the reactor
\end{enumerate}

\subsection{First Monte-Carlo}

We attempted to loop the single example `n' times but ran into an issue with creating many variables

This was solved by turning the positions and velocities into a (2, 3) array

\subsubsection{Issues}

\begin{enumerate}
    \item The neutrons still did not always spawn inside the reactor
    \item Probability density functions were not equipped to be used as a random input to another function
    \item Loops were too slow for large `n'
    \item Escape tests extremely slow
\end{enumerate}

\subsection{Solving Issues}

Speed: Neutrons were processed in a 3d array, removing the need for most loops

Probability density functions tweaked to allow mapping into other variables (screenshot pls this was such an annoying function to write)

Position: By taking the random variable (0, 1) as inputs and converting to spherical coordinates, the radius could always be kept within a sphere of 1 (screenshot this too)

Escape: Testing final position (radius) sped up significantly using Frobenius norm \(\sqrt{x^2+y^2+z^2}\)

\section{Extra Additions}

\end{document}